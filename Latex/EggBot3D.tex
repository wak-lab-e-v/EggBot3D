\newcommand{\Title}{EggBot aus dem 3D Drucker} 
\title{\Title}

\documentclass[a4paper]{article}
\usepackage{geometry}
\geometry{a4paper,tmargin=25mm,bmargin=35mm,lmargin=20mm,rmargin=20mm,footskip=5mm}
\usepackage[utf8x]{inputenc}
%\usepackage[T1]{fontenc}
\usepackage{fontspec}
\usepackage{graphicx}
\usepackage{subcaption}
%\usepackage{picins}
\usepackage{capt-of}
\usepackage{lastpage}
\usepackage[ngerman]{babel}
\usepackage[colorlinks=true,linkcolor=black,urlcolor=black]{hyperref}
%\usepackage[headtopline,headsepline,footsepline,footbotline]{scrpage2}
%\usepackage[headtopline,headsepline]{scrpage2}
\usepackage[headsepline,footsepline]{scrlayer-scrpage}
\usepackage{multirow}
\usepackage{listings}            
% fuer Stichwortverzeichnis
\usepackage{makeidx}
%\usepackage[automark,headsepline,footsepline,plainheadsepline,plainfootsepline]{scrpage2}

\usepackage{xcolor,soul}
\usepackage{colortbl} % Farbige Tabellen

\usepackage{longtable}

\usepackage{tabularx}
%\setheadtopline{.5pt}
%\setheadsepline{.5pt}

\pagestyle{scrheadings}
\clearscrheadfoot

%%standard commands
\newcommand{\ltab}{\raggedright\arraybackslash} % Tabellenabschnitt linksbündig
\newcommand{\ctab}{\centering\arraybackslash} % Tabellenabschnitt zentriert
\newcommand{\rtab}{\raggedleft\arraybackslash} % Tabellenabschnitt rechtsbündig

\definecolor{LightGray}{rgb}{0.9,0.9,0.9}
\newcommand{\CItoprowcolor}{\rowcolor{LightGray}}


%%Kopfzeile
\ihead{{\textbf{\large \Title}}}
%\chead{\textbf{\large \title}}
\ohead{\raisebox{0.1\totalheight}{\includegraphics[width=0.15\textwidth]{pictures/wak-lab-LOGO.png}}}

%%Fu\ss zeile
\cfoot{\pagemark}
\ofoot{\today}

   

\begin{document}
\maketitle
\large
%\tableofcontents
\begin{center}
%\includegraphics[height=10cm]{pictures/Ei.png}
\end{center}
%\tableofcontents
%\newpage
%\section*{Vorwort}

\newpage
\section{Einleitung}
\input{chapters/Einleitung}
\section{Programmierung}
\subsection{Install}
\begin{enumerate}
  \item Download the ARDUINO IDE (v 1.8.1 or above) here: https://www.arduino.cc/en/Main/Software and install it.
  \item Run the software. Select the Arduino/Genuino UNO(native USB port) board and the right COM PORT in the menu ,,tools->board``...
  \item Open and Upload the Ejjduino\_M0.ino code. CLICK HERE TO DOWNLOAD IT (decompress all the files inside the same folder, name it ,,Ejjduino\_M0``).
  \item 
\end{enumerate}

\subsection{Bewegungstest}
Ich habe aus der mit Python 2.7 vorbereitete Inkscape Installation die Dateien
\begin{itemize}
\item ebb\_serial.py
\item ebb\_motion.py
\item inkex.py
\end{itemize}
herausgesucht und mit ,,EggTest.py`` einen kleinen Bewegungstest programmiert. Da der Arduino noch kurz ,,hi`` sagt und  ca. 2 Sekunden braucht bist er sich mit seiner Firmwareversion meldet habe ich noch eine verzögerung von 2 Sekunden eingebaut.\\


\subsection{Serial Commands}
\begin{minipage}[t]{1.0\textwidth}
\captionof{table}{Hier die Tabelle}
\begin{tabular}{|p{8,1cm}|p{8,1cm}|}
\hline
\CItoprowcolor \textbf{Command} & \textbf{Description}\\
\hline
v & sendVersion\\
\hline
EM & enableMotors\\
\hline
SC & stepperModeConfigure\\
\hline
SP & setPen\\
\hline
SM & stepperMove\\
\hline
SMQB & stepperMoveQueryButton composite function enabling smooth movement\\
\hline
SE & ignore\\
\hline
TP & togglePen\\
\hline
PO & Engraver command\\
\hline
NI & nodeCountIncrement\\
\hline
ND & nodeCountDecrement\\
\hline
SN & setNodeCount\\
\hline
QN & queryNodeCount\\
\hline
SL & setLayer\\
\hline
QL & queryLayer\\
\hline
QP & queryPen\\
\hline
QB & queryButton\\
\hline
\end{tabular}
\label{tab:Tabelle1}
\end{minipage}




\subsection{Inkscape}
Die Pytonfunktionen des eggbot liegen in :inkscape Sphere-o-bot\textbackslash share\textbackslash extensions\\
 
\section{Inkscape}
\url{https://wiki.evilmadscientist.com/Installing_software}

\begin{enumerate}
  \item         Download the software in this ZIP archive (8 MB), and unzip it (ad-ink). (Your computer may unzip the archive automatically for you.)
  \item         Locate and open your Inkscape user extensions directory. You can find this location as follows:\begin{itemize}
  \item[A:]               Open Inkscape
  \item[B:]               From the menu, select Edit > Preferences
  \item[C:]               Select System from the list on the left panel of the preferences window.
  \item[C:]               On the right-hand panel is a list of directory locations. Click the ,,Open`` button to the right of the ,,User extensions`` folder location.
  \end{itemize}
  \item         Copy the contents of the ZIP archive (approximately 33 items including 2 subdirectories) into your Inkscape User extensions directory and relaunch Inkscape. Make sure that you copy all of these files, not just some of the files, and not just a folder containing these files.
  \item         Launch Inkscape, or quit and re-launch it if it is open.
\end{enumerate}

\subsection{Linux}
\subsubsection{inkscape}
Installation:
\begin{verbatim}
sudo add-apt-repository ppa:inkscape.dev/stable
sudo apt-get update
sudo apt-get install inkscape
\end{verbatim}

\subsubsection{arduino}
\url{https://linoxide.com/how-to-install-arduino-ide-on-ubuntu-20-04/}
\begin{verbatim}
mkdir arduino
cd arduino/
wget https://downloads.arduino.cc/arduino-1.8.15-linux64.tar.xz
tar -xvf ./arduino-1.8.15-linux64.tar.xz
cd arduino-1.8.15/
sudo ./install.sh
\end{verbatim}

Damit der Nutzer unter Linux mit dem Eggbot sprechen kann, muss er zur Gruppe: ,,dailout`` hinzugefügt werden.\\
\begin{verbatim}
ls -l /dev/ttyACM*
id
sudo usermod -a -G dialout <username>
\end{verbatim}
\section{AxiDraw Linux Commandline}
\url{https://axidraw.com/doc/cli_api/}
\begin{verbatim}
python -m pip install https://cdn.evilmadscientist.com/dl/ad/public/AxiDraw_API.zip
\end{verbatim}
\end{document}


